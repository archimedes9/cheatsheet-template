\documentclass[%
  a4paper,     % paper size
  landscape,   % landscape format
  english,     % language of the document
  7pt,         % font size
  fleqn        % left-align display math formulas
]{scrartcl}
% encoding
\usepackage[utf8]{inputenc}
% prettier fonts
\usepackage[T1]{fontenc}
% use palatino font fam­ily
\usepackage{mathpazo}
% language stuff
\usepackage{babel}

%%
% customize lists
%%
\usepackage{enumitem}
% make all lists more compact
\setlist{leftmargin=*,nosep}
% set label of all levels of itemize lists to bullet point
\setlist[itemize]{label=\textbullet}

%%
% customize page margins
%%
\usepackage[left=5mm,right=5mm,top=5mm,bottom=5mm]{geometry}

% no indendation of new paragraphs
\setlength{\parindent}{0pt}

% tables
\usepackage{tabularx}

% useful ams packages for math stuff
\usepackage{amsmath,amsthm,amsfonts,amssymb}

% overwrite fonts of headings and description lists
\setkomafont{disposition}{\normalfont\bfseries}
\setkomafont{descriptionlabel}{\normalfont\bfseries}

% multi-column layout
\usepackage{multicol}
% width of seperation rule between columns
\setlength{\columnseprule}{.5pt}

\begin{document}

\begin{multicols*}{%
  4 % number of columns to use
}

\section*{Usage}

This is a template for compact cheat sheets. Read the code for customization options.

\subsection*{Headings}

Use the familiar \texttt{section}, \texttt{subsection}, \texttt{subsubsection} or their asterisk variants to outline your sheet.

\subsection*{Math mode}

Both inline formulas, like $\frac{1}{n-1} \sum_{i=1}^n (x_i - \mu)^2$, and display formulas, like
\[
  \frac{1}{n-1}\sum_{i=1}^n (x_i - \mu)^2
\]
work. We included the \texttt{amsmath} package, so e.\,g. \texttt{align} also works
\begin{align*}
  \mathbb E(X) &= \frac{1}{n} \sum_{i=1}^n x_i\\
  \operatorname{Var}(X) &= \frac{1}{n} \sum_{i=1}^n x_i^2
\end{align*}

\subsection*{Lists}

We use the \texttt{enumitem} package to make \texttt{itemize} lists more compact.

\begin{itemize}
  \item Item 1
  \item Item 2
  \begin{itemize}
    \item Item 2.1
    \item Item 2.2
    \begin{itemize}
      \item Item 2.2.1
      \item Item 2.2.2
    \end{itemize}
  \end{itemize}
  \item Item 3
\end{itemize}

\subsection*{Tables}

Use \texttt{tabularx} to make tables of full column width.

\begin{tabularx}{\columnwidth}{|c|X|}
  \hline
  1 & Entry 1\\
  2 & Entry 2\\
  \hline
\end{tabularx}


\end{multicols*}

Veronika, Angela

\end{document}
